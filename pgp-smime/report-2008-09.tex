% $Id: report-2008-09.tex 3642 2008-09-25 07:55:43Z joostvb $
% $URL: svn+ssh://nagy/data/vc/svn/trunk/doc/nlnet-mailman-pgp-smime/report-2008-09.tex $
%
% Copyright © 2008 Joost van Baal <info.ad1810.com>
%

\documentclass[a4]{article}
\usepackage{a4wide}
\usepackage{url}

\title{First Secure List Server bi-monthly project report}

\author{Joost van Baal \url{<joostvb.ad1810.com>}}

\begin{document}

\maketitle

%\begin{abstract}
%foo
%\end{abstract}

\setlength{\parindent}{0pt}
\setlength{\parskip}{1.5ex}

\section*{Introduction}

As agreed upon in the June 13, 2008 document ``Memorandum of Understanding
Secure List Server Project'', this report documents work done by the author for
the Secure List Server project
(\url{http://non-gnu.uvt.nl/mailman-pgp-smime/}), as funded by the NLnet
Foundation.  It also lists the current plans for the project.

\section{Completed tasks}

Work on the project was started on 2008-06-24.  This was ahead of schedule:
the start was planned for 2008-07-01.

Here's a condensed overview of the progress made thus far.

\begin{tabular}{lll}
 Task                            & Planned  & Delivered   \\ \cline{1-3}
 Announce the project            & 08-07-15 & 08-07-03    \\
 Create a bzr repository         & 08-07-15 & 08-06-25    \\
 Merge the latest patch update   & 08-08-01 & 08-06-25    \\
 Port patch to latest release    & 08-08-01 & 08-07-26    \\
 Discuss Auditor's report        & 08-08-15 & 08-08-09    \\
 Publish first project report    & 08-09-01 & 08-08-31    \\
\end{tabular}

The project was announced in \url{Message-ID:
<20080703145455.GQ12960@bruhat.mdcc.cx>}, posted Thu, 3 Jul 2008 on the Mailman
Developers list
(\url{http://www.mail-archive.com/mailman-developers%40python.org/msg11056.html})
as well as on the PGP/SMIME Mailman devel list
(\url{http://ulm.ccc.de/pipermail/ssls-dev/2008-July/000019.html}).

The public Bazaar Revision Control repository is available from
\url{https://code.launchpad.net/~joostvb/mailman/2.1-pgp-smime}.

A patch for the latest Mailman release is available from
\url{http://non-gnu.uvt.nl/pub/mailman/mailman-2.1.11-pgp-smime_2008-07-26.patch.gz}.

Some extra time was spent on giving user support (using the PGP/SMIME Mailman
devel list) and setting up a test system.

\section{Planned tasks}

\begin{tabular}{lll}
 Task                            & Planned  \\ \cline{1-2}
 Bug: Implement test suite       & 08-12-15 \\
 Publish second project report   & 08-11-01 \\
 Bug: Enforce confidentiality    & 08-12-15 \\
 Bug: Better user interface      & 08-12-15 \\
 Publish third project report    & 09-01-01 \\
 Write and publish documentation & 09-01-15 \\
 Create a package of SLS         & 09-03-01 \\
 Publish fourth project report   & 09-03-01 \\
 Disseminate results             & 09-03-01 \\
 Try get SLS shipped w/ distros  & 09-03-01 \\
 Act upon auditors final report  & 09-04-01 \\
 Fifth and final project report  & 09-04-01 \\

\end{tabular}

The 3 Bug-tasks are the critical open bugs as found by Security Auditor Guus
Sliepen and published in ``Security Audit of the Secure List Server, Part I'',
August 1, 2008.  While fixing these bugs, the patch will be kept as
non-intrusive and minimal as possible.

The implementation of a test suite currently is in progress.

The documentation will be written for users, for list admins, for site admins,
as well as for developers.

Both a Debian and an RPM package for SLS will get build and published.

The dissemination of results will be done by announcing the achievements on
relevant mailing lists, and by giving a presentation at e.g. the FOSDEM
conference (will probably happen february 2009 in Brussels).

In order to get SLS shipped with Free Software operating system distibutions,
maintainers of Mailman packages for e.g. GNU/Linux distributions will get asked
(and offered help) to include the patch.  The author will work with the Debian
Mailman package maintainer to try to get the patched Mailman shipped with
Debian and Ubuntu, as discussed in a private conversation with the maintainer,
Tilburg, 2008-06-06.  The decision on wether or not to include this patch is
under control of the package maintainer (not the patch author).

\end{document}

