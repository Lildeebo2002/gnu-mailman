% $Id: report-2009-03.tex 4103 2009-03-18 11:58:46Z joostvb $
% $URL: svn+ssh://nagy/data/vc/svn/trunk/doc/nlnet-mailman-pgp-smime/report-2009-03.tex $
%
% Copyright © 2008, 2009 Joost van Baal <info@ad1810.com>
%

\documentclass[a4]{article}
\usepackage{a4wide}
\usepackage{url}

% \title{First Secure List Server bi-monthly project report\\September 1, 2008}
\title{Fourth Secure List Server bi-monthly project report}
% Jan 1, 2009

\author{Joost van Baal \url{<joostvb@ad1810.com>}}

\begin{document}

\maketitle

\setlength{\parindent}{0pt}
\setlength{\parskip}{1.5ex}

\section*{Introduction}

As agreed upon in the June 13, 2008 document ``Memorandum of Understanding
Secure List Server Project'', this report documents work done by the author for
the Secure List Server project
(\url{http://non-gnu.uvt.nl/mailman-pgp-smime/}), as funded by the NLnet
Foundation.  It also lists the current plans for the project.

This document is a follow-up to the Third Secure List Server bi-monthly
project report, as sent to Valer Mischenko on January 26, 2009.

%downloads, list traffic, commits

\section{Completed tasks}

Here's a condensed overview of the progress made thus far.

\begin{tabular}{lll}
 Task                            & Planned  & Delivered   \\ \cline{1-3}
 (Start project)                 & 08-07-01 & 08-06-24    \\
 (Milestone 1)                   & 08-08-15 & 08-08-09    \\
 (Milestone 2)                   & 08-12-15 & 09-01-06    \\
 Publish third project report    & 09-01-01 & 09-01-26    \\
 Write and publish documentation & 09-01-15 & 09-01-12    \\
 Create a package of SLS         & 09-03-01 & in progress \\
 Publish fourth project report   & 09-03-01 & 09-03-18    \\
 Disseminate results             & 09-03-01 & 09-02-08    \\
 (Milestone 3)                   & 09-03-01 & in progress \\
\end{tabular}

See the previous reports for details on the tasks completed for
Milestones 1 and 2.

Some documentation has been written and published.  The article ``Secure List
Server: Mailman, PGP and S/MIME - Support for encryption and authentication for
the GNU mailing list software'' from january 2009 is maintained using the SLS
version control system, and available from
\url{http://non-gnu.uvt.nl/mailman-pgp-smime/pgp-smime/talk/mailman-pgp-smime-talk.txt}.
This article was printed on paper and handed out at the 2 presentations in Ulm
and Brussels.  This documentation still needs to get integrated with the
documentation which comes with Mailman itself.  The current documentation
focusses on list and site admins; user oriented documentation still needs to be
written.

A lightning talk request for fosdem has been submitted on 08-11-06 and got
accepted, see \url{http://fosdem.org/2009/schedule/events/secure_list_server}.
The fosdem conference (see \url{http://fosdem.org/}) took place February 2009
at ULB Campus Solbosch in Brussels.  The talk (
http://fosdem.org/2009/node/164) took place on sunday February 8th, 10:20, and
lasted for 15 minutes.  The talk has been recorded on video; see
\url{http://ftp.heanet.ie/mirrors/fosdem-video/2009/lightningtalks/}.  (At
fosdem, the author also organised the PGP KeySigning Party, see
\url{http://fosdem.org/2009/keysigning}.)

Futhermore, a talk request for the Chaosseminar in Ulm has been submitted
08-11-24.  The talk took place at 09-01-12, see
\url{http://ulm.ccc.de/ChaosSeminar/2009/01_Mailman_PGP_SMIME}.  The talk has
been recorded on video; see
\url{http://ftp.ccc.de/regional/ulm/chaosseminar/200901-mailman/}.  About 17
Free Software developers (featuring Stefan Schlott, the original SLS-patch
author) attended the talk, which started at 20:00.  A lively discussion went on
till 21:30.  More information about this talk is in the Third Secure List
Server report.

All talk recordings are available from
\url{http://non-gnu.uvt.nl/mailman-pgp-smime/pgp-smime/talk/} also.

Some extra time was spent on keeping our code in sync with upstream by merging
it.  Feedback has been given to users on the GPG/SMIME Mailman development
list.

(Also, a Debian package for the Small Sister client was build; available from
\url{http://mdcc.cx/tmp/SmallMailClient/}.)


\section{Planned tasks}

Current plans are:

\begin{tabular}{lll}
 Task                            & Planned  \\ \cline{1-2}
 Create a package of SLS         & 09-03-01 \\
 (Milestone 3)                   & 09-03-01 \\
 Act upon auditors final report  & 09-04-01 \\
 Try get SLS shipped w/ distros  & 09-04-15 \\
 Fifth and final project report  & 09-04-15 \\
 (Milestone 4)                   & 09-04-15 \\
\end{tabular}

I'll also contact Mailman developer Barry Warsaw and ask him to perform his
review he has offered via the Mailman Developers list.

Both a Debian and an RPM package for SLS will get build and published.

Currently, Guus Sliepen is working on his final security audit report.

In order to get SLS shipped with Free Software operating system distibutions,
maintainers of Mailman packages for e.g. GNU/Linux distributions (and the
Sabayon and Smallsister projects) will get asked (and offered help) to include
the patch.  The author will work with the Debian Mailman package maintainer to
try to get the patched Mailman shipped with Debian and Ubuntu, as discussed in
a private conversation with the maintainer, Tilburg, 2008-06-06.  The decision
on wether or not to include this patch is under control of the package
maintainer (not the patch author).

\end{document}

