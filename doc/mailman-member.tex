% Complete documentation on the extended LaTeX mark up used for Python
% documentation is available in ``Documenting Python'', which is part
% of the standard documentation for Python.  It may be found online
% at:
%
%     http://www.python.org/doc/current/doc/doc.html

\documentclass{howto}

%  This is a template for short or medium-size Python-related documents, 
% mostly notably the series of HOWTOs, but it can be used for any
% document you like.   

% The title should be descriptive enough for people to be able to find
% the relevant document. 
\title{GNU Mailman - List Member Manual}

% Increment the release number whenever significant changes are made.
% The author and/or editor can define 'significant' however they like.
\release{0.03}
% CHANGELOG
% 0.03 Proofreading changes 
% 0.02 Proofreading changes
%   - proofread by Margaret McCarthy and Jason Walton 
% 0.01 First draft of document

% At minimum, give your name and an email address.  You can include a
% snail-mail address if you like.
\author{Terri Oda}
\authoraddress{terri(at)zone12.com}

\date{\today}        % XXX update before tagging release!
\release{2.1}        % software release, not documentation
\setreleaseinfo{}    % empty for final release
\setshortversion{2.1}      % major.minor only for software

\begin{document}
\maketitle

% This makes the Abstract go on a separate page in the HTML version;
% if a copyright notice is used, it should go immediately after this.
%
\ifhtml
\chapter*{Front Matter\label{front}}
\fi

% Copyright statement should go here, if needed.
% ...

% The abstract should be a paragraph or two long, and describe the
% scope of the document.
\begin{abstract}
\noindent
This document describes the list member interface for GNU
Mailman 2.1.  It contains instructions for subscribing, unsubscribing, 
viewing the archives, editing user options, getting password reminders,
and other subscriber-level tasks.  It also answers some common questions
of interest to Mailman list members.
\end{abstract}

\tableofcontents

% ============================================================================
\section{Introduction}

This document is intended to help the members of a Mailman 2.1 mailing list
learn to use the features available to them.  It covers the use of the
web and email interfaces for subscribing and unsubscribing, changing 
member options, getting password reminders and other subscriber-level
tasks.  It also answers some common questions of interest to Mailman list
members.

Information for list and site administrators is provided in
other documents.

This document need not be read in order.  If you are simply looking for
an answer to a specific question, jump to the appropriate place and 
references to other sections will be provided if necessary or potentially 
helpful.

\note{For the purposes of this document, 
we assume that the reader is familiar with common terms related to email (eg:
Subject line, body of the message) and web sites (eg: drop-down box, button) or
can look them up.  We also assume that the reader can already use his or her
email program and web browser well enough that instructions such as "send email
to this address" or "visit this web page" or "fill in the form provided" are
clear.  If you are not familiar with these actions, you may want to consult
other documentation to learn how to do these things with your particular
setup.} 

% ----------------------------------------------------------------------------
\subsection{Acknowledgements}

Sections of this document have been borrowed from the List Administrator Manual
found in Mailman CVS, which was written by Barry A. Warsaw, and from the in-line
help for Mailman 2.1.  

The rest of this manual has been written by Terri Oda.
Terri has been maintaining mailing lists since the year she attained 
voting age in Canada, although the two are not related.  She currently 
oversees the mailing lists at Linuxchix.org, as well as several smaller 
servers.  In the world outside of list administration, Terri is doing 
work with an artificial life spam detector, and is actually more of a 
programmer than technical writer.

Proofreading thanks go to Margaret McCarthy and Jason Walton.

%WRITEME: More here.  Do we need a license statement here?

% ----------------------------------------------------------------------------
\subsection{What is a mailing list?}

A mailing list is simply a list of addresses to which the same information
is being sent.  If you were a magazine publisher, you would have a list of
the mailing addresses of all the subscribers to the magazine.  In the case
of an electronic mailing list, we use a list of email addresses from people
interested in hearing about or discussing a given topic.  

Two common types of email mailing lists are announcement lists and discussion 
lists.  

Announcement lists are are used so that one person or group can send
announcements to a group of people, much like a magazine publisher's mailing
list is used to send out magazines.  For example, a band may use a mailing list
to let their fan base know about their upcoming concerts.


A discussion list is used to allow a group of people to discuss topics amongst
themselves, with everyone able to send mail to the list and have it distributed
to everyone in the group.  This discussion may also be moderated, so only
selected posts are sent on to the group as a whole, or only certain people are
allowed to send to the group.  For example, a group of model plane enthusiasts
might use a mailing list to share tips about model construction and flying.

Some common terms:
\begin{itemize}
	\item A "post" typically denotes a message sent to a mailing list.  
	(Think of posting a message on a bulletin board.)
	\item People who are part of an electronic mailing list are usually called 
	the list's "members" or "subscribers."  
	\item  "List administrators" are the people in charge of maintaining that 
	one list.  Lists may have one or more administrators.
	\item A list may also have people in charge of reading posts and deciding
	if they should be sent on to all subscribers.  These people are called
	list moderators.
	\item Often more than one electronic mailing list will be run using the same
	piece of software.  The person who maintains the software which runs
	the lists is called the "site administrator."  Often the site administrator
	also administrates individual lists.
\end{itemize}
% ----------------------------------------------------------------------------
\subsection{GNU Mailman}

GNU Mailman is software that lets you manage electronic mailing lists. It
supports a wide range of mailing list types, such as general discussion
lists and announce-only lists.  Mailman has extensive features which make it
good for list subscribers, such as easy subscription and unsubscription, 
privacy options, and the ability to temporarily stop getting posts from the 
list.  The list member features are covered in this document.

Mailman also has many features which make it attractive to list and site 
administrators.  These features are covered in the list and site administrator
manuals.

% ============================================================================
\section{Translating from our examples to real lists}

Often, it's easier to simply give an example than explain exactly how
to find the address for your specific list.  As such, we'll frequently 
give examples for a fictional list called 
\email{LISTNAME@DOMAIN} whose list information page can be found at
\url{http://WEBSERVER/mailman/listinfo/LISTNAME}.  

Neither of these are
real addresses, but they show the form of a typical list address.
The capital letters used for the list-specific parts of each address should 
make it easier to see what should be changed for each
list. Although specific list configurations may be different, you will
probably be able to just replace the words given in capital letters with the
appropriate values for a real list:

\begin{description}
	\item [LISTNAME] The name of your list.
	\item [DOMAIN] The name of the mail server which handles that list. 
	\item [WEBSERVER] The name of the web server which handles the list web interface.  This may be the same as DOMAIN, and often refers to the same machine, but does not have to be identical. 
\end{description}

As a real-life example, if you are interested in the mailman-users list, you'd 
make the following substitutions: LISTNAME=mailman-users, DOMAIN=python.org, WEBSERVER=mail.python.org.  As such, for the 
\email{mailman-users@python.org}
mailing list, the list information page can be found at the URL
\url{http://mail.python.org/mailman/listinfo/mailman-users}.  (These, unlike 
most of the examples given in this document, are real addresses.)

Most lists will have this information stored in the \mailheader{List-*} 
headers.  Many
mail programs will hide these by default, so you may have to choose to view
full headers before you can see these informational headers.  

% ============================================================================
\section{Mailman's interfaces}
Mailman has two different interfaces for the list subscriber: the web 
interface and the email interface.  Most discussion list subscribers use 
the email interface, since this includes the email address you use to send 
mail to all the subscribers of that list.

The interface you use for changing options is largely
a matter of preference, since most (but not all) of the options which can 
be changed from the web interface can also be changed by email.
 Usually it is easier to use the web interface for 
changing options, since the web interface provides instructions as you go, but
there are times when people may prefer the email interface, so both are 
provided.

% ----------------------------------------------------------------------------
\subsection{The web interface\label{sec:web}}
The web interface of Mailman is its selling point for many administrators,
since it makes it much easier for subscribers and administrators to see 
which options are available, and what these options do.  

Every mailing list is also accessible by a number of web pages.  Note that
the exact URLs are configurable by the site administrator, so they may be
different than what's described below.  We'll describe the most common
configuration, but check with your site administrator or hosting
service for details.

\begin{description}
	\item [List information (listinfo) page]
		\begin{itemize}
			\item Usually found at \url{http://WEBSERVER/mailman/listinfo/LISTNAME}
			 (for example, \url{http://lists.example.com/mailman/listinfo/mylist})

			\item The listinfo page is the starting point for the subscriber
			interface.  As one would assume from the name it's given, it 
			contains information about the LISTNAME list.
			Usually all the other subscriber pages can be accessed from this
			point, so you really only need to know this one address.

		\end{itemize}

	\item [Member options page]
		\begin{itemize}
			\item Usually found at \url{http://WEBSERVER/mailman/options/LISTNAME/EMAIL}
			(For example, \url{http://lists.example.com/mailman/options/mylist/kathy@here.com})

			\item This page can also be accessed by going to the listinfo page
			and entering your email address into the box beside the button
			marked "Unsubscribe or Edit Options" (this is near the bottom of the
			page).

			\item The member options page allows you to log in/out and change your 
			list settings, as well as unsubscribe or get a copy of your password
			mailed to you. 

			\item \textbf{To log in to your member options page}: 
			If you are not already logged in, there will be a box near the top for
			you to enter your password.  (If you do not know your password, see 
			Section~\ref{sec:getpassword} for more information on getting your 
			password.)  Enter your password in the box and press the button.

			\item Once you are logged in, you will be able to view and change
			all your list settings.

		\end{itemize}
	\item [List Archives] 
		\begin{itemize}
			\item Usually found at \url{http://WEBSERVER/pipermail/LISTNAME} if the 
			list is publicly archived, and 
			\url{http://WEBSERVER/mailman/private/LISTNAME} if the list is privately
			archives.  (For example, 
			\url{http://lists.example.com/pipermail/mylist} or
			\url{http://lists.example.com/mailman/private/mylist})

			\item The list archive pages have copies of the posts sent to the
			mailing list, usually grouped by month.  In each monthly group, the
			posts are usually indexed by author, date, thread, and subject.

			\item \note{Pipermail is the name of the default archiver that
			comes with Mailman.  Other archive programs are available.}

			\item If the archive is private, you will need to supply your
			subscribed email address and your password to log in.  (See
			Section~\ref{sec:getpassword} for more information on getting
			your password.)
		\end{itemize}
\end{description}

% ----------------------------------------------------------------------------
\subsection{The email interface\label{sec:email}}
Every mailing list has a set of email addresses to which messages can be 
sent.  There's always one address for posting messages to the list, one
address to which bounces are sent, and addresses for processing email
commands.  For a fictional mailing list called
\email{mylist@example.com}, you'd find these addresses:

\begin{itemize}
\item \email{mylist@example.com} -- this is the email address people should
      use for new postings to the list.

\item \email{mylist-join@example.com} -- by sending a message to this address,
      a new member can request subscription to the list.  Both the
      \mailheader{Subject} header and body of such a message are
      ignored.  Note that mylist-subscribe@example.com is an alias for
      the -join address.

\item \email{mylist-leave@example.com} -- by sending a message to this address,
      a member can request unsubscription from the list.  As with the
      -join address, the \mailheader{Subject} header and body of the
      message is ignored.  Note that mylist-unsubscribe@example.com is
      an alias for the -leave address.

\item \email{mylist-owner@example.com} -- This address reaches the list owner
      and list moderators directly.  This is the address you use if 
		you need to contact the person or people in charge.

\item \email{mylist-request@example.com} -- This address reaches a mail robot
      which processes email commands that can be used to set member
      subscription options, as well as process other commands.
      A list of members' email commands is provided in 
		Appendix~\ref{a:commands}.

\item \email{mylist-bounces@example.com} -- This address receives bounces from
      members whose addresses have become either temporarily or
      permanently inactive.  The -bounces address is also a mail robot
      that processes bounces and automatically disables or removes
      members as configured in the bounce processing settings.  Any
      bounce messages that are either unrecognized, or do not seem to
      contain member addresses, are forwarded to the list
      administrators.

\item \email{mylist-confirm@example.com} -- This address is another email
      robot, which processes confirmation messages for subscription
      and unsubscription requests.
\end{itemize}

There's also an -admin address which also reaches the list administrators,
but this address only exists for compatibility with older versions of
Mailman.

For changing options, we use the \email{LISTNAME-request}
address (for example, \email{mylist-request@example.com}).

Commands can appear in the subject line or the body of the message.  Each 
command should be on a separate line.  If your mail program automatically
appends a signature to your messages, you may want to put the word
"\var{end}" (without the quotes) on a separate line after your other commands. 
The \var{end} command tells Mailman not to process the email after that
point.

The most important command is probably the "\var{help}" command, since it 
makes Mailman return a message full of useful information about the
email commands and directions to the web interface.

Quick references to the subscriber commands have been provided in 
Appendices \ref{a:commands} and \ref{a:options}. (These have been slightly
adapted from the output of the \var{help} command.)


% ============================================================================
\section{I need to talk to a human!\label{sec:human}}

If you have any trouble with any of these commands, you can always reach the
person or people in charge of a list by using the list administrator email address. 
The list administrators can help you figure out
how to do something, subscribe/unsubscribe you, or change your 
settings if you are unable to change them yourself for some reason.  Please
remember that many mailing list administrators are volunteers who are donating
their spare time to run the list, and they may be very busy people.

This list administrator email address is in the form \email{LISTNAME-owner@DOMAIN}, where LISTNAME is the name of the list (eg: mailman-users) and DOMAIN is
the name of the server (eg: python.org).  
This email address, 
along with the email addresses of specific administrators, is given on the 
bottom of the list information pages. See Section~\ref{sec:web} for more 
information on finding the list information page for your list

% ============================================================================
\section{Subscribing and unsubscribing}
Since subscribing (joining) and unsubscribing (leaving) lists are often the 
only things a list member needs to know, these can both be done without 
requiring you to know a password.

% ----------------------------------------------------------------------------
\subsection{How do I join a list? (subscribe)\label{sec:subscribe}}

There are two common ways you can subscribe to a Mailman mailing list. 

Using the web interface:
\begin{enumerate}
	\item Go to the list information page for the list you want to join. 
	(This will probably be something like 
	\url{http://WEBSERVER/mailman/listinfo/LISTNAME})
	\item Look for the section marked "Subscribing to LISTNAME" and fill in the 
	boxes.  You can fill in the following:
	\begin{itemize}
		\item You \emph{must} enter your email address.
		\item You may choose to supply your real name.
		\item You may choose a password.   If you do not choose one, Mailman will
		 generate one for you.  

		\warning{Do NOT use a valuable password, since this 
		password may be mailed to you in plain text.  }
		\item If the list supports more than one language, you may be able to 
		choose your preferred language.  \note{This setting does not affect
		posts to the list, only pre-prepared Mailman texts such as your member
		options page.}
	\end{itemize}
	\item Press the subscribe button.  A new page should appear telling you 
	that your request has been sent.
\end{enumerate}

Using the email interface:
\begin{enumerate}
	\item Open a mail program which sends mail from the address you want to
	subscribe.
	\item Send a mail to the list subscription address, which will be in the 
	form \email{LISTNAME-join@DOMAIN}.  The subject and body
	of the message will be ignored, so it doesn't matter what you put there.
\end{enumerate}

After following one of these sets of instructions (you don't need to do
both!), there are a few possible outcomes depending upon the settings for
that list. 
\begin{itemize}
	\item You may receive an email message asking for confirmation that you 
really want to be subscribed to the list.  This is to prevent anyone from
subscribing you to lists without your permission.  Follow the instructions
given in the message to confirm your wish to be subscribed.
	\item A moderator may also need
to confirm your subscription if you are subscribing to a limited list.  
	\item Or
you may have to wait for a moderator \textit{and} follow the instructions in
the confirmation mail.
\end{itemize}

Once this is done, you will likely receive another message welcoming you to
the list.  This message contains some useful information including your list
password and some quick links for changing your options, so you may want to
save it for later reference.

\note{Subscribing can be done in other ways as well.  See 
Appendix~\ref{a:commands} for more advanced email subscribing commands.}

% ----------------------------------------------------------------------------
\subsection{How do I leave a list? (unsubscribe)\label{sec:unsubscribe}}

Don't want to be on a list any more?  If you're just going on vacation or
are too busy to read mails and want to temporarily turn them off, you may want
to stop mail delivery rather than unsubscribing.  This means you keep your
password and other settings so you can, for example, still have access to 
private list archives.  If this is what you'd
prefer, see Section~\ref{sec:nomail} for instructions on disabling mail
delivery temporarily.

If you actually want to leave the list, there are two common ways you can
unsubscribe from a Mailman mailing list.

Using the web interface:
\begin{enumerate}
	\item Go to the list information page for the list you want to leave.
	(This will probably be something like 
	\url{http://WEBSERVER/mailman/listinfo/LISTNAME})
	\item Look for the section marked "LISTNAME subscribers" (usually found 
	near the bottom of the page).
	\item There should be a button marked "Unsubscribe or Edit Options."  
	Enter your email address in the box beside this button and press the
	button.
	\item You should be brought to a new page which has an "Unsubscribe"
	button.  Press it to unsubscribe and follow the instructions given.
\end{enumerate}

Using the email interface:
\begin{enumerate}
	\item Open a mail program which sends mail from the address you want to 
	unsubscribe.
	\item Send a mail to the list unsubscribe address, which will be of the form
	\email{LISTNAME-leave@DOMAIN}.  
	The subject and body
	of this message will be ignored, so it doesn't matter what you put there.
\end{enumerate}

After following one of these sets of instructions (you don't need to do
both!), you will be sent a confirmation mail and must follow the
instructions given in that mail to complete the unsubscription.  This is to
stop people from unsubscribing you without your permission.  In addition, a
moderator may need to approve your unsubscription.

If you do not receive this confirmation mail with instructions, make sure
that you typed your email address correctly (if you were using the web 
interface to unsubscribe) and that the address you tried
to unsubscribe is, indeed, actually subscribed to that list.  For security
reasons, Mailman generates the same member options page regardless of
whether the address entered is subscribed or not.  This means that people
cannot use this part of the web interface to find out if someone is
subscribed to the list, but it also means that it's hard to tell if you just
made a typo.

Once your unsubscription has been processed, you will will probably receive
another message confirming your unsubscription from the list, and at that
point you should stop receiving messages.

If you wish to skip the confirmation process (for example, you might be
unsubscribing an address which no longer works), it is possible to bypass it by
using your password instead and either logging in to your options page using
it (See Section~\ref{sec:web}), or sending it with your email commands to
LISTNAME-request (See Appendix~\ref{a:commands} for advanced email 
unsubscription commands). See Section~\ref{sec:getpassword} for more 
information on getting your password.

% ============================================================================
\section{Passwords\label{sec:password}}
Your password was either set by you or generated by Mailman when you
subscribed. 
You probably got a copy of it in a
welcome message sent when you joined the list, and you may also receive a
reminder of it every month.  It is used to verify your identity to Mailman
so that only the holder of the password (you!) and the administrators
can view and change your settings.  

\warning{Do NOT use a valuable password for Mailman, since it can be
sent in plain text to you.}

% ----------------------------------------------------------------------------
\subsection{How do I get my password?\label{sec:getpassword}}
If you've forgotten your password and haven't saved the welcome message or 
any reminder messages, you can always get a reminder through the web interface:

\begin{enumerate}
	\item Go to the list information page for the list from which you wish to 
	get your password
	(This will probably be something like 
	\url{http://WEBSERVER/mailman/listinfo/LISTNAME})
	\item Look for the section marked "LISTNAME subscribers" 
	(this section is usually found near the bottom of the page).
	\item There should be a button marked "Unsubscribe or Edit Options."  
	Enter your email address in the box beside this button and press the
	button.
	\item You should be brought to a new page which has an "Password 
	Reminder" section.  Press the "Remind" button to have your password
	emailed to you.
\end{enumerate}

If you do not receive the password reminder email after doing this, make sure 
that you typed your
email address correctly and that the address you used is, indeed, actually
subscribed to that list.  For security reasons, Mailman generates the same
member options page regardless of whether the address entered is subscribed
or not.  This means that people cannot use this part of the web interface to
find out if someone is subscribed to the list, but it also means that it's
hard to tell if you just made a typo.

You can also get a reminder using the email interface, 
\begin{enumerate}
	\item Send a mail to \email{LISTNAME-request@DOMAIN} with the command 
	\var{password}

	Commands can appear
	in either the body or the subject of the message.  (See 
	Section~\ref{sec:email} for more information about sending mail
	commands.)  

	If you are not sending mail from your subscribed address, you can also
	specify this address by sending the command \nolinebreak{\var{password~address=$<$ADDRESS$>$}}.
\end{enumerate}

% ----------------------------------------------------------------------------
\subsection{How do I change my password?}
		\warning{Do NOT use a valuable password, since this 
		password may be mailed to you in plain text.  }

From the web interface:
\begin{enumerate}
	\item Log in to your member options page.  (See Section~\ref{sec:web} for 
   instructions on how to do this.)

	\item Look for the password changing boxes on the right-hand side of the
	page and enter your new password in the appropriate boxes, then press the
	button marked "Change My Password."
\end{enumerate}

This can also be changed for multiple lists at the same time if you are subscribed to 
more than one list on the same domain.  See Section~\ref{sec:global} for 
information about changing settings globally.


From the email interface:
\begin{enumerate}
	\item Send a mail to \email{LISTNAME-request@DOMAIN} with the command 
	\nolinebreak{\var{password~$<$OLDPASSWORD$>$~$<$NEWPASSWORD$>$}}.  

	Commands can appear
	in either the body or the subject of the message.  (See 
	Section~\ref{sec:email} for more information about sending mail
	commands.)  

	If you are not sending mail from your membership address, you can also
	specify this address with \var{address=$<$ADDRESS$>$} after $<$NEWPASSWORD$>$.
	
	For example, if \email{kathy@here.com} wanted to change her \var{mylist}
	password from \var{zirc} to \var{miko}, but she was sending mail from 
	her work address \email{kathy@work.com}, she could send a message
	to \email{mylist-request@example.com} with the subject set to 
	\nolinebreak{\var{password~zirc~miko~address=kathy@here.com}}.
\end{enumerate}

% ----------------------------------------------------------------------------
\subsection{How do I turn password reminders on or off? (reminders option)}
If you don't wish to the reminder email including your password every month, 
you can disable it from the member options page.  (You can always get the
password mailed out when you actually want it.  See 
Section~\ref{sec:getpassword} for instructions.)

Using the web interface:
\begin{enumerate}
	\item Log in to your member options page.  (See Section~\ref{sec:web} for 
	instructions on how to do this.)
	\item Look for the section marked "Get password reminder email for this
	list?" and change the value accordingly.
\end{enumerate}

This can also be changed for multiple lists at the same time if you are subscribed to 
more than one list on the same domain.  See Section~\ref{sec:global} for 
information about changing settings globally.

Using the email interface:
\begin{enumerate}
   \item Send a mail to \email{LISTNAME-request@DOMAIN} with the command
   \var{set~reminders~on} or \var{set~reminders~off}.

   Commands can appear
   in either the body or the subject of the message.  (See
   Section~\ref{sec:email} for more information about sending mail
   commands.)
	\item Set it to "on" to receive reminders, and "off" to stop receiving
	reminders.
\end{enumerate}


% ============================================================================
\section{Changing mail delivery}
% ----------------------------------------------------------------------------
\subsection{How do I turn mail delivery on or off? 
	(delivery option)\label{sec:nomail}}

You may wish to temporarily stop getting messages from the 
list without having to unsubscribe.
If you disable mail delivery, you will no longer receive messages, but will
still be a subscriber and will retain your password and other settings.  

This can be handy in a many different cases.  For example, you could be 
going on vacation or need a break from the list because you're too busy to 
read any extra mail. 
Many mailing lists also allow only subscribers to post to the list, so if you
commonly send mail from more than one address (eg, one address for at home
and another for when you're travelling), you may want to have more than
one subscribed account, but have only one of them actually receive mail.
You can also use this as a way to read private archives even on a list which
may be too busy for you to have sent directly to your mailbox.  All you need to do is subscribe, disable mail delivery, and use your password and email to 
log in to the archives.

To disable/enable mail delivery using the web interface:
\begin{enumerate}
	\item Log in to your options page. (See Section~\ref{sec:web} for instructions.)
	\item Go down to the section marked "Mail delivery" and select "Disabled"
		to stop receiving mail, and "Enabled" to start receiving mail.
\end{enumerate}

This can also be changed for multiple lists at the same time if you are subscribed to 
more than one list on the same domain.  See Section~\ref{sec:global} for 
information about changing settings globally.

To disable/enable mail delivery using the email interface:
\begin{enumerate}
   \item Send a mail to \email{LISTNAME-request@DOMAIN} with the command
   \var{set~delivery~off} or \var{set~delivery~on}.

   Commands can appear
   in either the body or the subject of the message.  (See
   Section~\ref{sec:email} for more information about sending mail
   commands.) 
	\item Set it to "off" to stop receiving posts, and "on" to start
	receiving them again.
\end{enumerate}

% ----------------------------------------------------------------------------
\subsection{How can I avoid getting duplicate messages? (duplicates option)
\label{sec:nodupes}}

Mailman can't completely stop you from getting duplicate messages, but it 
can help.  One common reason people get multiple copies of a mail is that 
the sender has used a "group reply" function to send mail to both the list and 
some number of individuals.  If you want to avoid getting these messages,
Mailman can be set to check and see if you are in the \mailheader{To} or
\mailheader{CC} lines of the message.  If your address appears there,
then Mailman can be told not to deliver another copy to you.

To turn this on or off using the web interface:
\begin{enumerate}
	\item Log in to your member options page.  (See Section~\ref{sec:web} 
	for more details on how to do this.)
	\item Scroll down to the bottom of the page to the section marked
	"Avoid duplicate copies of messages?" and change the value accordingly.
\end{enumerate}

This can also be changed for multiple lists at the same time if you are subscribed to 
more than one list on the same domain.  See Section~\ref{sec:global} for 
information about changing settings globally.

To turn this on or off using the email interface:
\begin{enumerate}
   \item Send a mail to \email{LISTNAME-request@DOMAIN} with the command
   \var{set~duplicates~on} or \var{set~duplicates~off}.

   Commands can appear
   in either the body or the subject of the message.  (See
   Section~\ref{sec:email} for more information about sending mail
   commands.)
	\item Set it to "on" to receive list copies of messages already sent
	to you, set it to "off" to avoid receiving these duplicates.
\end{enumerate}


% ----------------------------------------------------------------------------
\subsection{How do I change my subscription address?\label{sec:changeaddress}}
To change your subscription address, 
\begin{enumerate}
	\item Log in to your member options page.  (See Section~\ref{sec:web} 
	for more details on how to do this.)
	\item In the section marked "Changing your LISTNAME membership information,"
	enter your new address.
	\item If you wish to change your address for all subscriptions using the 
	old address, select the "Change globally" box.  If you have subscriptions
	under another address or for lists on a different domain, these will have
	to be done separately. See Section~\ref{sec:global} for more
	information about changing settings globally.
\end{enumerate}

There is no special way to do this from the email interface, but you can 
subscribe and unsubscribe for more or less the same effect.  (See 
Sections~\ref{sec:subscribe} and \ref{sec:unsubscribe} for more information
on subscribing and unsubscribing.)

% ----------------------------------------------------------------------------
\subsection{How do I stop or start getting copies of my own posts? (myposts
	option)\label{sec:getown}}
By default in Mailman, you get a copy of every post you send to the list.  
Some people like this since it lets them know when the post has gone through
and means they have a copy of their own words with the rest of a discussion, 
but others don't want to bother downloading copies of their own posts.

\note{This option has no effect if you are receiving digests.}

You may also want to see Section~\ref{sec:getack}, which discusses 
acknowledgement emails for posts sent to the list.

To set this using the web interface:
\begin{enumerate}
	\item Log in to your member options page.  (See Section~\ref{sec:web} 
	for more details on how to do this.)
	\item Look for the section marked "Receive your own posts to the list?"  
	Set it to "Yes" to receive copies of your own posts, and  "No" to avoid
	receiving them.
\end{enumerate}

To set this using the email interface:
\begin{enumerate}
   \item Send a mail to \email{LISTNAME-request@DOMAIN} with the command
   \var{set~myposts~on} or \var{set~myposts~off}.

   Commands can appear
   in either the body or the subject of the message.  (See
   Section~\ref{sec:email} for more information about sending mail
   commands.)
	\item Set it to "on" to receive copies of your own posts, and "off"
	to avoid receiving them.
\end{enumerate}


% ----------------------------------------------------------------------------
\subsection{How can I get Mailman to tell me when my post has been received
by the list? (ack option)\label{sec:getack}}

On most lists, you will simply receive a copy of your mail when it has gone
through the list software, but if this is disabled (See 
Section~\ref{sec:getown}), your list mail delivery is disabled (See
Section~\ref{sec:nomail}), you are not subscribed to that topic (See
Section~\ref{sec:sometopic}) or you
simply want an extra acknowledgement from the system, this option may
be useful to you.

\note{If you are not subscribed to the list, this option cannot be used.  
You must either check the archives yourself (if the list has public archives),
ask someone who is subscribed to the list, or subscribe to use this option.}

To set this using the web interface:
\begin{enumerate}
	\item Log in to your member options page.  (See Section~\ref{sec:web} 
	for more details on how to do this.)
	\item Look for the section marked "Receive acknowledgement mail when you
	send mail to the list?"  
	Set it to "Yes" to receive a mail letting you know your post has been
	received, and  "No" to avoid receiving such an acknowledgement.
\end{enumerate}

To set this using the email interface:
\begin{enumerate}
   \item Send a mail to \email{LISTNAME-request@DOMAIN} with the command
   \var{set~ack~on} or \var{set~ack~off}.

   Commands can appear
   in either the body or the subject of the message.  (See
   Section~\ref{sec:email} for more information about sending mail
   commands.)
	\item Set it to "on" if you wish to receive mail letting you know your 
	post has been received, and  "off" to avoid receiving such an 
	acknowledgement.
\end{enumerate}

% ----------------------------------------------------------------------------
\subsection{I don't seem to be getting mail from the lists.  What should
	I do?}
There are a few common reasons for this:
\begin{itemize}
	\item No one has sent any mail to the list(s) you're on for a little while.

	To check if this is the case, try visiting the archives of the list 
	(assuming that the list has archives).  If the list has no archives, you
	may have to ask another subscriber.  (See Section~\ref{sec:web} for help
	in finding the list archives.)

	\note{Generally, it is considered 
	impolite to send test messages to the entire list.
	If you feel a need to test that the list is working and for some reason you
	cannot simply compose a regular message to the list, it is less disruptive 
	to send a help message to
	the list request address (LISTNAME-request@DOMAIN) to see if that works,
	or to contact the list 
	administrator (LISTNAME-owner@DOMAIN) to ask if the list is working.}

	\item You were bouncing mail and have had mail delivery (temporarily) 
	disabled by the list software.

	If your mail provider "bounces" too many messages (that is, it tells
	Mailman that the message could not be delivered)
	Mailman eventually stops trying to send you mail.  This feature allows
	Mailman to gracefully handle addresses which no longer exist (for example,
	the subscriber has found a new internet service provider and forgot to 
	unsubscribe the old address), as well
	as addresses which are temporarily out-of-service (for example, the 
	subscriber has used up all of the allotted space for his or her email
	account, or the subscriber's mail provider is experiencing difficulties).

	Even if you are unaware of any difficulties with your mail provider, it 
	is a good idea to check this.  Some popular webmail providers and 
	internet servers are not as reliable as one might assume, nor is the
	internet as a whole.  You may want to also send yourself a test message
	from another account or ask a friend to send you a test message to make
	sure your subscribed address is working.

	To check if this may be the reason you are not receiving messages, log in 
	to the your options page (See
	Section~\ref{sec:web} for more details on how to do this) and 
	look at your options.  There should be one marked "Mail Delivery" --
	if it is set to "Disabled," set it to "Enabled" to start receiving mail
	again. (For more instructions on disabling or enabling mail delivery,
	see Section~\ref{sec:nomail}.)

	\note{Even if you have not been disabled at the time you check, you could be
	bouncing messages and not have reached the threshold for your 
	subscription to be disabled.  You may need to check again.}

	\item There is a delay or break in the networks between you and the 
	list server.

	No matter what many of us would like, the internet is not 100\% 
	reliable, nor is it always fast.  Sometimes, messages simply take a long 
	time to get to you.  Try to be patient, especially if the server is far 
	(in terms of networks, not geography, although often one implies the other)
	from your internet service provider.

	To check if this might be causing your problem, you can try pinging
	the list server or tracing the route between you and it.  (Instructions
	on how to do this varies from platform to platform, so you may want to 
	use a search engine to find those appropriate for you.)

	\item The Mailman installation on the list server is not functioning or
	not functioning properly.
	
	To test if this is a case, try visiting the list's web interface and 
	try sending a message to \email{LISTNAME-request@DOMAIN} with the command 
	"\var{help}" (without the quotes) in the \mailheader{Subject}.  If 
	neither of these works after a reasonable length of time, this may be
	the problem.  You may wish to contact either the list or site 
	administrator(s).
\end{itemize}


% ============================================================================
\section{Digests}
% ----------------------------------------------------------------------------
\subsection{How can I start or stop getting the list posts grouped into one 
big email? (digest option)\label{sec:digest}}

Groups of posts are called "digests" in Mailman.  Rather than get messages
one at a time, you can get messages grouped together.  On a moderately busy
list, this typically means you get one email per day, although it may be
more or less frequent depending upon the list.

You may also want to look at Section~\ref{sec:MIME} which discusses MIME
and plain text digests.

To turn digest mode on or off using the web interface,
\begin{enumerate}
	\item Log in to your member options page.  (See Section~\ref{sec:web} 
	for more details on how to do this.)
	\item Look for the section marked "Set Digest Mode."  

	Set it to "On" to 
	receive messages bundled together in digests.  Set it to "Off" to 
	receive posts separately.
\end{enumerate}

To turn digest mode on or off using the email interface,
\begin{enumerate} 
   \item Send a mail to \email{LISTNAME-request@DOMAIN} with the command
   \var{set~digest~plain} or \var{set~digest~mime} or \var{set~digest~off}.
   
   Commands can appear
   in either the body or the subject of the message.  (See
   Section~\ref{sec:email} for more information about sending mail
   commands.)
   \item Set it to "off" if you wish to receive individual posts separately,
	and to "plain" or "mime" to receive posts grouped into one large mail.
	See Section~\ref{sec:MIME} for more information on plain versus MIME
	digests.
\end{enumerate}


% ----------------------------------------------------------------------------
\subsection{What are MIME and Plain Text Digests?  How do I change which one
I get? (digest option)\label{sec:MIME}}

MIME is short for Multipurpose Internet Mail Extensions.  It is used to 
send things by email which are not necessarily simple plain text.  (For 
example, MIME would be used if you were sending a picture of your dog to 
a friend.)  

A MIME digest has each message as an attachment inside the message, along
with a summary table of contents.  

A plain text digest is a simpler form of digest, which should be readable 
even in mail readers which don't support MIME.  The messages are simply put 
one after the other into one large text message.

Most modern mail programs do support MIME, so you only need to choose 
plain text digests if you are having trouble reading the MIME ones.

\note{This option has no effect if you are not receiving mail bunched
as digests.  (See Section~\ref{sec:digest} for more information on 
receiving mail as digests.)}

To set your digest type using the web interface:
\begin{enumerate}
   \item Log in to your member options page.  (See Section~\ref{sec:web}
   for more details on how to do this.)
   \item Look for the section marked "Get MIME or Plain Text Digests?."

	Set it to "MIME" to receive digests in MIME format, or "Plain text" to 
	receive digests in plain text format. 
\end{enumerate}

This can also be changed for multiple lists at the same time if you are subscribed to 
more than one list on the same domain.  See Section~\ref{sec:global} for 
information about changing settings globally.

To set your digest type using the email interface,
\begin{enumerate}
   \item Send a mail to \email{LISTNAME-request@DOMAIN} with the command
   \var{set~digest~plain} or \var{set~digest~mime}.

   Commands can appear
   in either the body or the subject of the message.  (See
   Section~\ref{sec:email} for more information about sending mail
   commands.)
   \item Set it to "plain" to get posts bundled into a plain text digest,
	or "mime" to get posts bundled together into a MIME digest.
\end{enumerate}

% ----------------------------------------------------------------------------
\section{Mailing list topics\label{sec:topics}}

Some lists are set up so that different topics are handled by Mailman.
For example, the courses list on Linuxchix.org is a discussion list for
courses being run by linuxchix members, and often there are several courses
being run at the same time.  
(eg: Networking for beginners, C programming, \LaTeX ~document mark up.)  
Each of the courses being run is a separate topic on the list so that people
can choose only to receive the course they want to take.

These
topics must be configured by the list administrator, but it is the
responsibility of each poster to make sure that their post is put with 
the correct topic.  Usually, this means adding a tag of some type to the 
subject line (eg: [Networking] What type of cables do I need?) or making
sure the \mailheader{Keywords} line has the right information.  (By default, 
you can put a \mailheader{Keywords} section in the beginning of the body
of your message, but this can be configured by your list administrator.)
Note that these tags are case-insensitive.

% ----------------------------------------------------------------------------
\subsection{How do I make sure that my post has the right 
	topic?\label{sec:posttopic}}

When a list administrator defines a topic, he or she sets three things:
\begin{itemize}
	\item a topic name
	\item a regular expression (regexp)
	\item a description
\end{itemize}

You can view this information by logging in to your member options page. 
 (See Section~\ref{sec:web} for more details on how to do this.) and 
clicking on the "details" link for any topic that interests you.

To post on a given topic, you need to make sure that the 
\mailheader{Keywords} or \mailheader{Subject} headers in a message  
match the \emph{regular expression} for that topic.  
Regular expressions can actually be fairly complex, so you may want to 
just ask the list administrator if you don't know how to make
heads or tails of the expression given.  

Most Mailman topic expressions will be fairly simple regular expressions, so 
in this document we will simply give you some common examples.  Regular 
expressions are a bit too complex to teach in a few lines here, so if you
really want to understand how the regular expressions work, you should
find a tutorial or reference elsewhere.  (For example, DevShed has a decent
tutorial at 
\url{http://www.devshed.com/Server_Side/Administration/RegExp/})  

Here are some examples of possible regular expressions and matching lines:

\begin{tableii}{l|l}{}{Regular expression}{Matching lines}
	\lineii{zuff}{Keywords: zuff}
	\lineii{zuff}{Keywords: ZUFF}
	\lineii{zuff}{Keywords: Zuff}
	\lineii{zuff}{Keywords: amaryllis, zuff, applesauce}
	\lineii{zuff}{Subject: [zuff] Do you have the right stuff for zuff?}
	\lineii{zuff}{Subject: Do you have the right stuff for zuff?}
	\lineii{zuff}{Subject: What is zuff?}
\hline
	\lineii{\textbackslash[zuff\textbackslash]}{Keywords: [zuff]}
	\lineii{\textbackslash[zuff\textbackslash]}{Subject: [zuff]	Do you have the right stuff?}
	\lineii{\textbackslash[zuff\textbackslash]}{Subject: Online zuff tutorials (was Re: [zuff] What is zuff?)}
\end{tableii}

A few notes:
\begin{itemize}
	\item The matching is case-insensitive, so if zuff matches, so will ZUFF,
	 zuFF, and any other variations in capitalization.
	\item Some characters have special meaning in a regular expression, so
	to match those characters specifically, they must be "escaped" with a 
	backslash (\textbackslash).  As you can see in the above example, 
	[ and ] are such characters.  (Others include ".", "?", and "*").  
	The backslash is also used for other things (I wasn't kidding about 
	regular expressions being complex: consult other documentation
	for details about other uses of the backslash character), but this 
	is the most likely use in a topic expression.
\end{itemize}

% ----------------------------------------------------------------------------
\subsection{How do I subscribe to all or only some topics on a 
	list?\label{sec:sometopic}}

If topics have been set up by your mailing list administrator, you can 
choose to subscribe to only part of a list by selecting the topics you
want to receive.

If you wish to get all messages sent to the list, make sure you 
are not subscribed to any topics.

\begin{enumerate}
	\item Log in to your member options page.  (See Section~\ref{sec:web} 
	for more details on how to do this.)
	\item Look for the section marked "Which topic categories would you like 
	to subscribe to?" 

	If any topics are defined, you can select those you wish.  If you do
	not select any topics of interest, you will receive all posts 
	sent to the list.
\end{enumerate}

You probably also want to look at Section~\ref{sec:notopic} which discusses
changing your settings for messages where no topic is set.

% ----------------------------------------------------------------------------
\subsection{How do I get or avoid getting messages with no topic set?
\label{sec:notopic}}
If you wish to get all messages sent to the list, make sure you are 
not subscribed to any specific topic.  (See Section~\ref{sec:sometopic}.)  

If you are only subscribed to some topics, you can either choose to either
receive or not receive messages with no topic set, much the way you can 
choose to subscribe only to certain topics.

To change this setting,
\begin{enumerate}
	\item Log in to your member options page.  (See Section~\ref{sec:web} 
	for more details on how to do this.)
	\item Look for the section marked "Do you want to receive message that do 
	not match any topic filter?"

	If you wish to receive messages with no topic set, select "Yes."  If you
	do not wish to receive such messages, choose "No."
\end{enumerate}

This setting has no effect if you are not subscribed to any topics.

% ============================================================================
\section{Setting other options}

% ----------------------------------------------------------------------------
\subsection{Change Globally?  Set Globally? What does that mean?
	\label{sec:global}}

For some of the options given in your member options page, there is a 
tick-box which says "Change Globally" or "Set Globally."  
This means that if you change this
option, you can also have the change made for all your other list 
subscriptions with the same address to lists on the same domain.  
This can be handy if, for example, you 
want to make sure all your passwords are the same, or you are going on 
vacation and want to turn off mail delivery from all the lists.

% ----------------------------------------------------------------------------
\subsection{How do I change my name as Mailman knows it?
	\label{sec:changename}}

To change your subscription name, 
\begin{enumerate}
	\item Log in to your member options page.  (See Section~\ref{sec:web} 
	for more details on how to do this.)
	\item In the section marked "Changing your LISTNAME membership information,"
	enter your new name in the appropriate box.
\end{enumerate}

This can also be changed for multiple lists at the same time if you are subscribed to 
more than one list on the same domain.  See Section~\ref{sec:global} for 
information about changing settings globally.

\note{You do not need to have a subscription name set.}

% ----------------------------------------------------------------------------
\subsection{How do I set my preferred language?}

Mailman is available with many different languages.
(For a complete listing see \url{http://mailman.sourceforge.net/i18n.html}.)  This means that, if your list has
other languages enabled, you may be able to have the web interface, etc. in a 
language of your choice.  

\note{This does NOT necessarily mean that all the posts sent to the list will 
be in the language you selected.  Only the pre-prepared texts presented by 
Mailman will be affected by this setting.  Posts are in whatever language the
poster uses.}

Your preferred language is set when you subscribe (see 
Section\ref{sec:subscribe}), and can be changed later if the list supports
more than one language.  

To change your preferred language in Mailman,
\begin{enumerate}
	\item Log in to your member options page.  (See Section~\ref{sec:web} for 
   instructions on how to do this.)
	\item Go to the section marked "What language do you prefer?" and choose
	the appropriate language from the drop-down list.  If there is no
	drop-down list of languages, the list you are on probably only supports
	one language.  
\end{enumerate}

If your list does not support the language you would prefer to use, you may
contact the list administrator (LISTNAME-owner@DOMAIN) to see if it can be 
added, but remember that this may mean some work that the list and/or site 
administrator(s) do not have time or the ability to do.  

If your language of choice is not available because no translation
exists for Mailman, please consider volunteering your time as a translator.
For more information you may want to consult the mailman-i18n mailing 
list at \url{http://mail.python.org/mailman/listinfo/mailman-i18n}.
(i18n is a common short-hand for "internationalization" because the word starts 
with an i, ends with an n, and has 18 letters in between.  If you mumble a bit,
i18n even sounds a bit like "internationalization.")

% ----------------------------------------------------------------------------
\subsection{How do I avoid having my name appear on the subscribers list?
	(the hide option)\label{sec:nolist}}

If you do not want to have your email address show up on the subscriber list
for any reason, you can opt to have it concealed.

Common reasons for doing this include avoiding unsolicited bulk email (spam).  
By default, the subscribers list is obscured to hinder spam harvesters,
but if you feel this is insufficient it's easy enough to remove address
from the subscriber list given in the information pages or by email request.
(Note that this does not conceal your address from the list administrators.)
You may wish to see Section~\ref{sec:antispam} for more information on what
Mailman can do to help avoid spam.  

To change this setting using the web interface:
\begin{enumerate}
	\item Log in to your member options page.  (See Section~\ref{sec:web} for 
   instructions on how to do this.)
	\item Go to the section marked "Conceal yourself from subscriber list?" and
	choose "Yes" to hide your name from the list, or "No" to allow your name
	to appear on the list.
\end{enumerate}

To change this setting using the email interface:
\begin{enumerate} 
   \item Send a mail to \email{LISTNAME-request@DOMAIN} with the command
   \var{set~hide~on} or \var{set~hide~off}.

   Commands can appear 
   in either the body or the subject of the message.  (See
   Section~\ref{sec:email} for more information about sending mail
   commands.)
   \item Set it to "on" to conceal your email address from the membership
	list, or "off" to stop concealing your address.
\end{enumerate}

% ============================================================================
\section{Other common questions}

% ----------------------------------------------------------------------------
\subsection{How do I view the list archives?}
If the list has archives, they can be viewed by going to a web page address.
This address usually linked from the list information page and can be found in 
the \mailheader{List-Archive} of every list message unless your list 
administrator has disabled these headers.  (Many mail programs hide the 
\mailheader{List-Archive} mail header, so you may have to tell your 
mail program to allow you to view full headers before you will be able to 
see it.)

Public archives usually have addresses of the form 
\url{http://WEBSERVER/pipermail/LISTNAME/} and private archives usually
have addresses of the form \url{http://WEBSERVER/mailman/private/LISTNAME}.

See Section~\ref{sec:web} for more information on finding the addresses of a
list.

% ----------------------------------------------------------------------------
\subsection{What does Mailman do to help protect me from unsolicited bulk email
(spam)?\label{sec:antispam}}

A technical list's archives may include answers to a range of
different questions.  Often, the people who have posted these answers would
be happy to help someone who doesn't quite understand the answer, and don't
mind giving their address out for that purpose.  But
although it would be wonderful if everyone could contact each other easily,
we also want to make sure that the list and list archives are not abused by 
people who send spam.  

To make a range of options available to list administrators, Mailman allows
a variety of configurations to help protect email addresses.  
Many of these settings are optional to the list administrator, so your
particular list may be set up in many ways.  List administrators 
must walk a fine line between protecting subscribers and making it difficult
for people to get in touch.  

\begin{itemize}
	\item Subscriber lists
		\begin{itemize}
			\item The list administrator can choose to have the subscriber list
			public, viewable only to list members, or viewable only to list 
			administrators.
			\item The subscriber list is shown with the addresses obscured to
			make it difficult for spam harvesters to collect your address.
			\item You can choose to have your address hidden from the subscriber
			list.  (See Section~\ref{sec:nolist} for more information.) 
			\item \note{The entire subscriber list is always available to the
			list administrators.}
		\end{itemize}

	\item List archives
		\begin{itemize}
			\item The list administrator can choose for the archives to be public,
			viewable only to members (private), or completely unavailable.
			\item The HTML archives which are created by Pipermail (the 
			archiving program which comes default with Mailman) contain only
			obscured addresses.  Other archiving programs are available and can
			do different levels of obfuscation to make addresses less readable.
			\item If you wish to be more sure, you can set the mail header
			\mailheader{X-no-archive} and Mailman will not archive your posts.
			\warning{This does not stop other members from quoting your posts,
			possibly even including your email address.}
		\end{itemize}

	\item Limited posting to the lists
		\begin{itemize}
			\item The list administrator can choose who can post to the list.
			Most lists are either moderated (a moderator or administrator
			reviews each posting), set so only subscribers may post to the list,
			or allow anyone to post to the list.
			\item By allowing only subscribers to post to a list, Mailman often 
			blocks all spam and some viruses from being sent through the list.
			As such, this is a fairly common setting used by list administrators.
		\end{itemize}

	\item Anonymous lists
		\begin{itemize}
			\item Lists can also be made fully anonymous: all identifying
			information about the sender is stripped from the header before the
			message is sent on.  
			\item This is not typically used for anti-spam measures (it has
			other uses), but it could be used in that way if desired.
		\end{itemize}
\end{itemize}

Of course, many address-obscuring methods can be circumvented by determined 
people, so be aware that the protections used may not be enough.

% ============================================================================
\appendix
% ----------------------------------------------------------------------------
\section{Email commands quick reference\label{a:commands}}
\begin{list}{}{}
	\item confirm $<$CONFIRMATION-STRING$>$
		\begin{list}{}{}
			\item
        Confirm an action.  The confirmation-string is required and should be
        supplied within a mailback confirmation notice.
		\end{list}

   \item end
		\begin{list}{}{}
			\item
        Stop processing commands.  Use this if your mail program automatically
        adds a signature file.
		\end{list}

   \item help
		\begin{list}{}{}
			\item
        Receive a copy of the help message.
		\end{list}

   \item info
		\begin{list}{}{}
			\item
        Get information about this mailing list.
		\end{list}

   \item lists
		\begin{list}{}{}
			\item
        See a list of the public mailing lists on this GNU Mailman server.
		\end{list}

    \item {password [$<$OLDPASSWORD$>$ $<$NEWPASSWORD$>$] [address=$<$ADDRESS$>$]}
		\begin{list}{}{}
			\item
        Retrieve or change your password.  With no arguments, this returns
        your current password.  With arguments $<$OLDPASSWORD$>$ and $<$NEWPASSWORD$>$
        you can change your password.
		\end{list}

    \item set ...
		\begin{list}{}{}
			\item
        Set or view your membership options.

        Use `set help' (without the quotes) to get a more detailed list of the
        options you can change.  This list is also given in 
        Appendix~\ref{a:options}.

        Use `set show' (without the quotes) to view your current option
        settings.
		\end{list}

    \item{subscribe [$<$PASSWORD$>$] [digest|nodigest] [address=$<$ADDRESS$>$]}
		\begin{list}{}{}
			\item
        Subscribe to this mailing list.  Your password must be given to
        unsubscribe or change your options, but if you omit the password, one
        will be generated for you.  You may be periodically reminded of your
        password.

        The next argument may be either: `nodigest' or `digest' (no quotes!).
        If you wish to subscribe an address other than the address you sent
        this request from, you may specify `address=$<$ADDRESS$>$' (no brackets
        around the email address, and no quotes!)
		\end{list}

    \item {unsubscribe [$<$PASSWORD$>$] [address=$<$ADDRESS$>$]}
		\begin{list}{}{}
			\item
        Unsubscribe from the mailing list.  If given, your password must match
        your current password.  If omitted, a confirmation email will be sent
        to the unsubscribing address. If you wish to unsubscribe an address
        other than the address you sent this request from, you may specify
        `address=$<$ADDRESS$>$' (no brackets around the email address, and no
        quotes!)
		\end{list}

    \item {who [$<$PASSWORD$>$] [address=$<$ADDRESS$>$]}
		\begin{list}{}{}
			\item
        See everyone who is on this mailing list.  The roster is limited to
        list members only, and you must supply your membership password to
        retrieve it.  If you're posting from an address other than your
        membership address, specify your membership address with
        `address=$<$ADDRESS$>$' (no brackets around the email address, and no
        quotes!)
		\end{list}
\end{list}

% ----------------------------------------------------------------------------
\section{Member options quick reference\label{a:options}}

\begin{list}{}{}
   \item set help
		\begin{list}{}{}
	        \item Show this detailed help.
		\end{list}

   \item {set show [address=$<$ADDRESS$>$]}
		\begin{list}{}{}
        \item View your current option settings.  If you're posting from an address
        other than your membership address, specify your membership address 
        with `address=$<$ADDRESS$>$' (no brackets around the email address, and no
        quotes!).
		\end{list}

   \item {set authenticate $<$PASSWORD$>$ [address=$<$ADDRESS$>$]}
		\begin{list}{}{}
        \item To set any of your options, you must include this command first, along
        with your membership password.  If you're posting from an address   
        other than your membership address, specify your membership address  
        with `address=$<$ADDRESS$>$' (no brackets around the email address, 
        and no quotes!).
		\end{list}

   \item set ack on\\
    set ack off
		\begin{list}{}{}
			\item
        When the `ack' option is turned on, you will receive an
        acknowledgement message whenever you post a message to the list.
		\end{list}

   \item set digest plain\\
    set digest mime\\
    set digest off
		\begin{list}{}{}
			\item
        When the `digest' option is turned off, you will receive postings
        immediately when they are posted.  Use `set digest plain' if instead
        you want to receive postings bundled into a plain text digest
        (i.e. RFC 1153 digest).  Use `set digest mime' if instead you want to
        receive postings bundled together into a MIME digest.
		\end{list}

   \item set delivery on\\
    set delivery off
		\begin{list}{}{}
			\item
        Turn delivery on or off.  This does not unsubscribe you, but instead
        tells Mailman not to deliver messages to you for now.  This is useful
        if you're going on vacation.  Be sure to use `set delivery on' when
        you return from vacation!
		\end{list}

   \item set myposts on\\
    set myposts off
		\begin{list}{}{}
			\item
        Use `set myposts off' to avoid receiving copies of messages you post to
        the list.  This has no effect if you're receiving digests.
		\end{list}

    \item set hide on\\
    set hide off
		\begin{list}{}{}
			\item
        Use `set hide on' to conceal your email address when people request
        the membership list.
		\end{list}

    \item set duplicates on\\
    set duplicates off
		\begin{list}{}{}
			\item
        Use `set duplicates off' if you want Mailman not to send you messages
        if your address is explicitly mentioned in the To: or Cc: fields of
        the message.  This can reduce the number of duplicate postings you
        will receive.
		\end{list}

    \item set reminders on\\
    set reminders off
		\begin{list}{}{}
        \item Use `set reminders off' if you want to disable the monthly password
        reminder for this mailing list.
		\end{list}
\end{list}

\end{document}
